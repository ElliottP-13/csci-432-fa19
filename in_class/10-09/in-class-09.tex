\documentclass{article}

\usepackage[margin=1in]{geometry}
\usepackage{enumerate}

\usepackage{amsfonts}
\def\R{{\mathbb R}}
\def\N{{\mathbb N}}

\title{In-Class Exercise 07}
\author{CSCI 432}
\date{09-12 October 2019}

\begin{document}
\maketitle

\noindent
Group Number:\\
Group members present today:

\section*{Amortized Analysis}

Reminder about readings: CLRS Chapter 17 and EPI Chapter 12

So far, we have talked about the following time complexity anyalysis techniques:
\begin{enumerate}
\item \emph{Worst-case analysis}: At worst, how much time will this algorithm take?
Line-by-line, we consider the worst-case behavior.  If the line is in a
loop, we multiply by the worst-case number of times that loop~iterates.
\item \emph{Average case analysis for randomized algorithms}: On average, how long
will this algorithm take to run?  Note: sometimes it may have bad
worst-case runtime.
\end{enumerate}
This week, we focus on:
\begin{enumerate}
\item[3.] \emph{Amortized analysis}: Rather than look at the worst-case per
line, consider the average-case per line.  This is similar to, but
distinct from, average case analysis.
\end{enumerate}

\begin{enumerate}[(1)]
\item Consider the following example: we have an stack $A$.  It
begins empty, then we call \texttt{A.push()} to add elements to it.
Suppose that we implement this with an array.  The naive approach would
be to create a new array each time \texttt{A.push()} is called that is
one unit longer than the previous array.  If the
cost of creating a new array is equal to the size of the new array, then the total cost after
$2$ step is $3$ ($1$ for the first push to create a new array of size
$1$ + $2$ for the second push to create a new array of size $2$).
What is the total cost after $4$ steps?  $5$ steps? $7$ steps? $n$ steps?
\vspace{1in}
\item Instead, suppose that we double the size of $A$ each time.  So, the
first push will create an array of size $1$ containing the single pushed
element.  The total cost so far is $1$.  When we push the second
element, we double the size of $A$ from $1$ to $2$.
The total cost so far is $3$ ($1$ for the first push + $2$ for the second
push).  What is the total cost after $4$ steps? $5$ steps? $7$ steps?
$n$ steps?
\vspace{1in}
\item For each of the implementations described above, what is the
\emph{average} cost per push?
\pagebreak
\item Consider a skyline of $n$ unit-width buildings.  We can be given the
skyline as an array $A$ of building heights.  We can say that a building
has a good view if either no building to the left of it is taller, or no
building to the right of it is taller.  Give a linear-time algorithm to
compute the set of buildings with a good view.
\end{enumerate}

\end{document}