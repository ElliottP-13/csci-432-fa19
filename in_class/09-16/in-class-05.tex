\documentclass{article}

\usepackage[margin=1in]{geometry}

\usepackage{amsfonts}
\def\R{{\mathbb R}}
\def\N{{\mathbb N}}

\title{In-Class Exercise 05}
\author{CSCI 432}

\begin{document}
\maketitle

\noindent
Group Number:\\
Group members present today:

\section*{Sample Offline Data}
Randomized algorithms can take various forms:
\begin{itemize}
    \item Uses randomization in order to have expected runtime better than
        worst-case runtime.
    \item Uses randomization to produce the correct answer or an approximation
        of it, with high
        probability.
    \item Takes a random number generator (often built into the programming
        language) and produces a random subset or random
        perturbation of the data.  These types of algorithms are often used to
        introduce randomness in the first two types of algorithms.
\end{itemize}
EPI 5.12 (Sample Offline Data) gives an algorithm to compute a random
subset of an array data.  Use the back of this page or another sheet of paper
for your answers.

\begin{enumerate}
    \item Write pseudocode for this algorithm using a while loop instead of a
        for loop.
    \item What is the decrementing function for the while loop?
    \item For the following questions, suppose the input array is
        $A=[1,2,\ldots,n]$, an let $S$ be the set
        returned. If $k=1$, what is the probability that $S=\{1\}$?  If $t$ is
        an integer between one and $n$ (inclusive), what is the
        probability that $S=\{t\}$?  Justify your answer.
    \item If $k=2$, what is the probability that $S=\{i,j\}$ for $i \neq j$
        integers between one and $n$ (inclusive)? Be careful, as
        $\{i,j\}$ should be considered as a set, not an ordered list.
    \item What is the decrementing function of your while loop?
    \item What is the total space complexity of this algorithm? Justify.
\end{enumerate}

\section*{Additional Probability Review}
If you finish the above, work on the following questions:
\begin{enumerate}
    \item In the algorithm above, if we set $k=n$ and return an array instead of
        a set, is our
        array a random permutation of the input?
    \item You might recall that the pigenhole principle is that if we have $n$
        boxes that contain $n+1$ objects, then one box must have at least two
        objects.  What is the expected number of objects per box?  Now, suppose
        we have $n$ boxes and $n$ objects.  What is the
        expected number of boxes with exactly one object?
    \item Shake-a-day is a gambling game, where you pay $\$0.50$ and roll five
        dice.  If all dice are the same number, then you win the pot (often, a
        portion stays behind to seed the next pot, but we will ignore that in
        our calculations).  What is the amount of money in the pot needed for
        your expected roll to have a positive value?
\end{enumerate}

\end{document}
