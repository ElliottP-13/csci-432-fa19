\documentclass[a4paper]{article}



\usepackage{../project}



% ========================================================================

\title{Shor's Algorithm}

\lhead{Elliott Pryor, Benjamin Bushnell, Shengnan Zhou}

\author{Elliott Pryor, Benjamin Bushnell, Shengnan Zhou}

\date{18 November, 2019}

% ========================================================================

\usepackage{multicol}

\setlength{\columnsep}{0cm}

\begin{document}



\maketitle % This line creates the title (DO NOT CHANGE)



For our $+1$ we've decided to walkthrough Shor's algorithm using a classical approach
 We will implement Shor's algorithm on a classical computer. This will show that Shor's algorithm works at factoring numbers
 The classical implementation will take longer than on a quantum computer, but the logic is the same on any computer
 The reason we chose this $+1$ is because we found it impossible to improve upon Shor's algorithm
 and unreasonable compare Shor's algorithm to other existing factoring methods as they all function on a classical computer
 Our $+1$ provides context for how Shor's algorithm functions on real world numbers
In this paper we will walk through an example to provide insight into how Shor's algorithm works
 This gives a high level overview of how the classical algorithm will work.



%TODO just added some, feel free to add or modify

\section{Introduction}

Shor's algorithm was invented in 1994, and it cleverly uses properties of quantum computing to quickly solve prime factorization of even large numbers
 This is exactly what RSA assumes is not possible. The short description of Shor's algorithm is that we can take any random guess $g$
 run the guess through Shor's algorithm, and get back a better guess $g^{p/2} \pm 1$
 For any large number $N$, when we take a random guess $g$, $mN = g^p +1$ for some $p$ and $m$
 Break this down, we get $(g^{p/2}+1)(g^{p/2}-1) = mN$. Once we have the two better guesses
 we can use Euclid's algorithm to find the shared factors, thus breaks the encryption
 However, the most important part of this algorithm is to find $p$, and it takes very very long time without quantum computing
 Quantum computing can take in multiple values as superpositions, and find $p$ fast


For sections \ref{classicalSec} and \ref{quantumSec} assume we are trying to find: 

$$ab = C$$

where $a$ and $b$ are factors of $N$ and $a,b,C \in \naturals$.



\section{Classical Computation}

\label{classicalSec}

The classical part of this algorithm is really simple. We just have to guess a random number as our factor
 First we run Euclid's algorithm on the guess to make sure that the GCD of the guess and $N$ is 1
 If the GCD is 1 then we know that our guess and N are co-prime, we then feed our guess into Shor's algorithm to improve our guess
 If the GCD of the guess and $N$ is not 1, then we guessed a factor (or multiple of a factor) and we do not need to use the quantum portion


\section{The Quantum Algorithm}

\label{quantumSec}



The quantum part of this algorithm is what turns the random number that we guessed into the actual factor
 In short, it does this by finding the period of some function that we can relate to the factors
 Given that that is a gross oversimplification of what happens, we will cover some of the math supporting Shor's Algorithm


First, we must state a few known relations. 



\textbf{Euler's Theorem}, as seen in Equation \ref{eulerTheorem}, provides a relation given two co-prime integers
 $a, n$. In Equation \ref{eulerTheorem}, $r$ represents the order of $a$ in the multiplicative group
 $r$ is the Euler-Totient function, but for our case we can consider it some integer
\begin{equation}

a ^ r \equiv 1 \textrm{(mod n)}

\label{eulerTheorem}

\end{equation}





Given $a^x = mN + r$ for $a,x,m,N,r \in \integers$. Then equation \ref{repeatingPower} holds for some $y
 p, k \in \integers$.

\begin{equation}

a^{x + yp} = kN + r

\label{repeatingPower}

\end{equation}



We know that the guess given to the quantum portion of this algorithm, $g$, does not share any factors with the number we are trying to factor
 $C$. Then by Euler's theorem (Theorem \ref{eulerTheorem}) we can express this as $r^p = mC + 1$
 We can rearrange this to $(r^{p/2} + 1)(r^{p/2} - 1) = mC$. Then we know that our factors of $C$ are related to $(r^{p/2} + 1)$ and $(r^{p/2} - 1)$
 Once we find these numbers, we can use Euler's formula to calculate the factors


In order to find $p$ we use theorem \ref{repeatingPower} and some properties of quantum computers
 By theorem \ref{repeatingPower}, we know that $g^{x + yp} = kC + r$ mod(C) = $r$ $\forall y \in \integers$
 So we can raise our guess to integer powers and search for when the remainder repeats
 Classically, we cannot do this efficiently. However, with quantum computers we have a superposition of states that we can exploit to efficiently compute this
 If we poll a random remainder value from our modulus calculation we get a superposition of the states that result in this value: $g^{x}$
 $g^{x + p}$, $g^{x+2p}$... This has a period of $p$ which we can by taking the Fourier transform of this superposition
 The Fourier transform returns the period of the function \cite{shor}. Assuming that $p$ is even
 then we are done! We have now found the factors of $C$. If $p$ is odd, then $r^{p/2}$ is not an integer
 and we have to restart the process with a new guess. 



\section{The +1 Outline}

In the $+1$ we implement Shor's algorithm on a classical computer. The main parts of this are:

\begin{enumerate}

    \item Euclid's Algorithm for GCD/finding the guess

    \item Period finding

\end{enumerate}



Euclid's algorithm for finding the greatest common divisor (GCD) has been known for centuries
 We will implement this algorithm in our final $+1$ solution. 



The second part is where the quantum computer would have significant advantages. We can find the period of the function classically
 but it will require more time. Once we find the period we can continue with Shor's algorithm as normal
 The period finding approach that we will use classically will involve checking each value until we detect repetition
 This will take time proportional to the period of the function. 



We will run this algorithm on several numbers to show that it successfully factors these numbers and walk through a couple of examples showing (in-detail) how the algorithm arrives at its conclusions


\section{Example}

To better understand how Shor's algorithm works, let's walk through an example.



Let's say $N=21$, and we want to find $a, b$ such that $N = a \cdot b$ and $a, b$ are prime numbers
First, we take a random guess $g=11$. We know that the GCD of 11 and 21 is 1, so we proceed to turn $g$ into a better guess $g^{p/2} \pm 1$
 Now we want to find $p$ such that $g^p = m \cdot N + 1$ for some $m$.



We can find $p$ by calculating the period at which $a ^{x+ yp}$ (mod N) repeats. On a quantum computer this is very fast
 however, we can easily do this by hand as 21 is a very small number (and our power is also small)
 Below, is 11 raised to the powers 0-19 (mod 21).



$$1, 11, 16, 8, 4, 2, 1, 11, 16, 8, 4, 2, 1, 11, 16, 8, 4, 2, 1, 11$$ 



We can see that the cycle repeats every 6 iterations. Therefore, we know that $p = 6$


We find $11^ 6 = 84360 \cdot 21 + 1$. Now we can have the better guess $g^{p/2} + 1 = a \cdot s$ for some factor $s$
 and $g^{p/2} - 1 = b \cdot t$ for some factor $t$. To find $a,b$, we need to use Euclid's algorithm to find the common factors
$g^{p/2} + 1 = 1332$, $gcd(1332, 21) = 3$. And $g^{p/2} - 1 = 1330$, $gcd(1330, 21) = 7$


Now we found the two prime factors of $N=21$ are $a = 3$ and $b = 7$.



\newpage

\bibliographystyle{abbrvnat}

\bibliography{citations}



\end{document}