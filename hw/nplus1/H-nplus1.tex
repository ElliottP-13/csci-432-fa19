% csci432, N+1 st homework
% Due LDOC

\documentclass{article}

\title{N+1 Homework Assignment}
\author{Elliott Pryor}
\date{9 December, 2019}

\begin{document}

\maketitle

%This is your final assignment for CSCI 432.

%Write a two-page paper describing to me how you have grown as a student,
%computer
%scientist, mathematician, engineer, or a researcher in this class, and more generally,
%in this
%semester.  To support your argument, you should include your
%homework or writing samples (or excerpts from them) in an appendix as
%evidence (and reference them!)

%If you do not feel that you've grown,
%explain why.

%Remember, style counts. Use complete sentences.

I have grown in a number of ways this semester. I am currently a Physics and Computer Science major;
however, I am planning on switching to Mathematics and Computer Science. Because these fields are very math-intensive,
my mathematical abilities have grown significantly. I significantly strengthened my ability to recognize different types of problems so that I can quickly find a solution.

The course puts a large emphasis in proving algorithm correctness. This has strengthened my ability to write formal proofs.
This forces me to look at each problem with a very critical eye and ask myself at each step why that step must be true.
This has been ingrained into my thought patterns so that I think of doing these proofs as I work through any mathematical process.
I often think of what the loop invariant will be when I create any loop and how this leads to the final solution.
This is very helpful when writing new algorithms as I think more in advance about how the pieces fit together and if my algorithm will produce the result that I anticipated.

%TODO add example

Thinking through proofs is also helpful when solving physics problems as I consider each algebraic step and ensure that it is mathematically correct.
This significantly reduces the number of 'silly' errors made while solving equations.
I convince myself that each step is right and once I know that every step is right I know that I have reached the right solution.
This allows me to be much more flexible in solving problems as I don't have to memorize a very specific equation for each problem,
but I can re-derive or modify more simple equations and laws to arrive at the solution.

Learning about different algorithm techniques have taught me to quickly recognize certain problem types and apply efficient approaches.
For example, I have learned when to apply divide and conquer, dynamic programming,
or greedy approaches. This significantly decreases the running time of the algorithms that I make as they are not brute force but instead apply more elegant techniques to speed it up.

Problem recognition also helps me come up with solutions to complicated problems.
Often the task seems very hard to compute at all (including brute force). But recognizing what kinds of problems fit into certain categories makes it easier to solve problems.
Sometimes it seems incredibly hard to figure out, but you can easily do it with a graph search problem or by dynamic programming.
This means I have much less wasted time trying to figure out how to approach the problem but can jump straight to the solution.

For example, in the ACM ICPC competition there was a problem that asked to find the fewest keystrokes to move between lines on a text editor.
Initially this seems very very complicated as you have to consider movements up,
down, left, and right but also you have to consider what happens if you go from a larger line to a smaller line and that you can wrap around the line.
I had no idea how to approach solving this problem, but it is just a shortest path algorithm on a graph.
If you represent the text file as a graph it is very easy to use breadth first search (or another search algorithm such as Djikstra's) to find the shortest path.
Recognizing what problems fit into what categories is a key component of computer science,
 and this course has given me more practice with recognizing specific approaches that will work.
 



\end{document}