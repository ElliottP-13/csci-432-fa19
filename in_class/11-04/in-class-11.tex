\documentclass{article}

\usepackage{algorithm,algpseudocode}
\usepackage[margin=1in]{geometry}
\usepackage{enumerate}
\usepackage{multicol}
\usepackage{graphicx}
\DeclareGraphicsExtensions{.pdf,.PDF,.png,.PNG}
\graphicspath{{figs/}}

\usepackage{amsfonts,amsmath}
\def\R{{\mathbb R}}
\def\N{{\mathbb N}}

\title{In-Class Exercise 10}
\author{CSCI 432}
\date{28 October 2019}

\begin{document}
\maketitle

\noindent
Group Number:\\
Group members present today:

\section*{Linear Programming}

\begin{multicols}{2}

\begin{enumerate}
\item On this graph paper, draw the following lines:
\begin{align}
4x_1 - x_2 &= 8\\
2x_1 + x_2 &= 10\\
5x_1 -2x_2 &= -2 \\
x_1 &= 0\\
x_2 &= 0
\end{align}
HINT: use the bottom
left corner as the point~$(0,0)$
\item Shade in the feasible space of points $(x_1,x_2)\in \R^2)$ that
satisfy the following~inequalities:
\begin{align}
4x_1 - x_2 &\leq 8\\
2x_1 + x_2 &\leq 10\\
5x_1 -2x_2 &\geq -2 \\
x_1 &\geq 0\\
x_2 &\geq 0
\end{align}
\item As dotted lines, draw the lines where $x_1+x_2=0$, $x_1+x_2=2$, and
$x_1+x_2=2$.  What do you notice about these lines?
\vspace{.75in}
\item Suppose we want to maximize $x_1+x_2$ subject to the constraints
above.  It turns out that the answer always lies on a vertex of the
space you just shaded!  Compute $f(x_1,x_2)=x_1+x_2$ for each of the
vertices. Which point attains the maximum value?
\vspace{.75in}
\item Write this linear program in standard form.\\
\vspace{2in}
\item What happens if I change the last constraint to~$x_1 \leq 0$?
\end{enumerate}

\end{multicols}

\includegraphics[width=\textwidth]{graph-paper}

\end{document}