\documentclass{article}
\usepackage{../fasy-hw}
\usepackage{ wasysym }

%% UPDATE these variables:
\renewcommand{\hwnum}{3}
\author{TODO-Your Group Number and Names Here}
\collab{TODO-list your collaborators here}
\date{due: 18 October 2019}

\begin{document}

\nextprob
If 23 people are in a room, then the probability that at least two of them have
the same birthday is at least one half.  This is known as the birthday paradox,
since the number 23 is probably much lower than you would expect.  How many
people do we need in order to have 50% probability that there are three people
with the same birthday?
As a reminder, when giving an algorithm as an answer, you
are expected to give:
\begin{itemize}
    \item A prose explanation of the problem and algorithm.
    \item Psuedocode.
    \item The decrementing function for any loop or recursion, or a runtime
        justification.
    \item Justification of why the runtime is linear.
    \item The loop invariant for any loops, with full justification.
\end{itemize}

\nextprob
Suppose we have a graph $G=(V,E)$ and three colors, and randomly assign a color
each node (where each color is equally likely).
\begin{enumerate}
    \item What is the probability that every edge has two different colors on
        assigned to its two nodes?
    \item What is the expected number of edges that have different colors
        assigned to its two nodes?
\end{enumerate}

\nextprob
CLRS, Question 15-6.

\nextprob
For the Greedy make change algorithm described in class on 10/02, describe the
problem and solution in your own words, including the use of pseudocode (with
more details than what was written in class).  Note: you do not need to give a
loop invariant and the proof of termination/runtime complexity.

\nextprob
Suppose we have $n$ items hat we want to put in a knapsack of capacity $W$.  The i-th item has
weight $w_i$ and value $v_i$.  The knapsack can hold a total weight of $W$ and
we want to maximize the value of the items in the knapsack.
The \emph{0-1 knapsack problem} will assign each item one of two states: in the
knapsack, or not in the knapsack.  The \emph{fractional knapsack problem} allows
you to take a percentage of each item.
\begin{enumerate}
    \item Give an $O(n\log n)$ greedy algorithm for the fractional knapsack problem.
    \item Give an $O(nW)$ time algorithm that uses dynamic programming to solve
        the 0-1 knapsack problem.
\end{enumerate}

\end{document}
