\documentclass{article}
\usepackage{../fasy-hw}
\usepackage{ wasysym }

%% UPDATE these variables:
\renewcommand{\hwnum}{1}
\author{Elliott Pryor}
\collab{None}
\date{due: 6 September 2019}

\begin{document}

\nextprob
Consider the third function defined in EPI, Section 13.1 (Compute the
Intersection of Two Sorted Arrays).
\begin{enumerate}
    \item When we design an algorithm, we design the algorithm to solve a
        problem or answer a question.  What is the problem that this algorithm
        solves?
    \item Prove that the while loop terminates using a decrementing function.
\end{enumerate}

\nextprob
Prove the following statement: Every tree with one or more nodes/vertices has
exactly $n-1$ edges.

\nextprob
Consider the following statement: If $a$ and $b$ are both odd numbers, then $ab$ is
an odd number.
\begin{enumerate}
    \item An odd number, n, is any number that can be expressed as $2a + 1$ st. $a \in \integers$
    \item An even number, n, is any number that can be expressed as $2a$ st. $a \in \integers$
    \item If $ab$ is an even number, then $a$ and $b$ are not both odd.
    \item Show that if $ab$ is an even number, then a and b are not both odd.
    	
    	\begin{proof} By Contradiction. \\
Assume not, assume that the product $ab$ is an odd number and both $a$ and $b$ 				are odd. Then, by the definition of an odd number, $a$ can be expressed as $2*n + 1$ for $n \in \integers$ and $b$ can be expressed as $2*m + 1$ for $m \in \integers$. Then by substitution $ab = (2*m + 1) * (2*n + 1) = 4mn + 2m + 2n + 1 = 2 *(2mn + m + n) + 1 = 2k + 1$ where $k = 2mn + m + n$, $k \in \integers$ because addition and multiplication are closed under the integers. \\
Therefore, $ab$ is odd by the definition of an odd number, a contradiction, and the claim is proven true by contradiction.
    	\end{proof}
    	
\end{enumerate}

\end{document}
