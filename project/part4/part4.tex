\documentclass[a4paper]{article}

\usepackage{../project}

% ========================================================================
\title{Shor's Algorithm}
\lhead{Elliott Pryor, Benjamin Bushnell, Shengnan Zhou}
\author{Elliott Pryor, Benjamin Bushnell, Shengnan Zhou}
\date{18 November, 2019}
% ========================================================================
\usepackage{multicol}
\setlength{\columnsep}{0cm}
\begin{document}


\maketitle % This line creates the title (DO NOT CHANGE)


\section{Example}
To better understand how Shor's algorithm works, let's walk through an example.

Let's say $N=21$, and we want to find $a, b$ such that $N = a \cdot b$ and $a, b$ are prime numbers.
First, we take a random guess $g=11$. We want to turn $g$ into a better guess $g^{p/2} \pm 1$. Now we want to find $p$ such that $g^p = m \cdot N + 1$ for some $m$.

After some calculations (this is where we need quantum computer to compute $p$), we find $11^ 6 = 84360 \cdot 21 + 1$. So $p=6$. Now we can have the better guess $g^{p/2} + 1 = a \cdot s$ for some factor $s$, and $g^{p/2} - 1 = b \cdot t$ for some factor $t$. To find $a,b$, we need to use Euclid's algorithm to find the common factors. 
$g^{p/2} + 1 = 1332$, $gcd(1332, 21) = 3$. And $g^{p/2} - 1 = 1330$, $gcd(1330, 21) = 7$. 

Now we found the two prime factors of $N=21$ are $a = 3$ and $b = 7$.





























\end{document}