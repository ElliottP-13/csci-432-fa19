\documentclass{article}

\usepackage{algorithm,algpseudocode}
\usepackage[margin=1in]{geometry}
\usepackage{enumerate}
\usepackage{multicol}
\usepackage{graphicx}
\DeclareGraphicsExtensions{.pdf,.PDF,.png,.PNG}
\graphicspath{{figs/}}

\usepackage{amsfonts,amsmath}
\def\R{{\mathbb R}}
\def\N{{\mathbb N}}

\title{In-Class Exercise 10}
\author{CSCI 432}
\date{28 October 2019}

\begin{document}
\maketitle

\noindent
Group Number:\\
Group members present today:

\section*{Concurrent Programming}

\begin{enumerate}
\item What is the difference between concurrency and parallelism? (Feel
free to use the internet if you are unsure).
\vspace{1in}


\item What are the possible return values of the following algorithm?  What
is the expected return value?
\begin{algorithm}\caption{\textsc{ComputeX}}
{\bf Input:} $\emptyset$\\
{\bf Output:} $x$, an integer
\begin{algorithmic}[1]
\State $x=0$
\For{PARALEL $i=1$ to $3$}
\State $x=x+1$
\EndFor\\
\Return $x$
\end{algorithmic}
\end{algorithm}
\vspace{1in}

\item Above is an  example of a \emph{race condition},
where running concurrent
threads could result in multiple outputs.  Explain an example
application where this could be problematic.
\vspace{3in}

\pagebreak
\item Consider the two problems on the Collatz conjecture (sections 12.13
and 19.9).  Test the Collatz conjecture for $n=5$.
\vspace{1in}

\item Give pseudocode for the parallel version of the Collatz check.
\vspace{3in}

\item The EPI book discusses the trade-off between the cost of starting
threads and the benefit for having things run concurrently (and
hopefully in parallel).  What do you conjecture to be the difference for
the optimal number of threads to be running in different programming
languages (for the same algorithm)?
\end{enumerate}

\end{document}