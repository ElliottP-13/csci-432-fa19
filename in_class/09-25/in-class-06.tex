\documentclass{article}



\usepackage[margin=1in]{geometry}

\usepackage{algorithm,algpseudocode}



\usepackage{amsfonts}

\def\R{{\mathbb R}}

\def\N{{\mathbb N}}



\title{In-Class Exercise 06}

\author{CSCI 432}



\begin{document}

\maketitle



\noindent

Group Number:\\

Group members present today:



\section*{Making Change}



\begin{enumerate}

    \item Give an example where making change using the greedy algorithm does

        not work.

        \vspace{1in}

    \item Write a dynamic program for making change using dynamic programming.

\end{enumerate}



\newpage

\section*{Bellman-Ford Algorithm}

\begin{algorithm}\caption{\textsc{Bellman-Ford}($G$, $\omega$, $v$)}\label{alg:seb}

    {\bf Input:} graph $G=(V,E)$, weight function $\omega \colon E \to \R$, and initial vertex $v \in V$ \\

    {\bf Output:} detects negative cycle if one exists.  Otherwise, returns

    shortest distance to every vertex in $G$ from $v$.\\

    \begin{algorithmic}[1]

        \State $dist \gets$ real-valued array of length $|V|$, and indexed by $V$
        \State $dist[v] \gets 0$

        \State $i \gets 1$



        \While{$i < n$}

            \For{$(u,w) \in E$}

                \State $dist(w) \gets \min \{ dist(w), dist(u)+ \omega(u,w) \}$

            \EndFor

            \State $i++$

        \EndWhile



        \For{$(u,w) \in E$}

            \If{ $dist(w) > dist(u)+ \omega(u,w)$  }\\

                \quad\quad\quad \Return ``Negative Cycle Detected"

            \EndIf

        \EndFor\\

        \Return $dist$

    \end{algorithmic}

\end{algorithm}



\begin{enumerate}

    \item Explain to each other, in words, (1) what is the problem; (2) how this

        algorithm works.  (No need to write down answer for this one).

    \item Work through a small example.

    \item What is the runtime?

    \item What is the loop invariant of each of the loops?

\end{enumerate}



\end{document}
