\documentclass{article}
\usepackage{../fasy-hw}
\usepackage{ wasysym }

%% UPDATE these variables:
\renewcommand{\hwnum}{4}
\author{Your Name Goes Here}
\collab{TODO-list your collaborators here}
\date{due: 18 November 2019}
\title{Homework 4 - Individual}

\begin{document}

\maketitle

This fourth homework assignment is due on 18 November 2019, and should be
submitted to BOTH Gradescope and D2L.  If you want feedback on the HW before the
final exam, please submit a draft of the hw to the H-04-FEEDBACK folder in
Gradescope.  Early submissions will be given feedback, but will not be graded.

If you are asked to come up with an algorithm, you are expected to give an
algorithm that beats the brute force (and, if possible, of optimal time
complexity).
With your algorithm, please provide the following:
\begin{itemize}
    \item A prose explanation of the problem and the algorithm.
    \item Psuedocode.
    \item The decrementing function for any loop or recursion, or a runtime
        justification.
    \item The loop invariant for any loops, with full justification.  Note: if
        loops are nested, you just need to explicitly give the loop invariant
        proof of the outer loop.
\end{itemize}

\nextprob
Find (or design) an example of a graph arising in data
with $4$--$10$ vertices, and at least five
edges.  Use the union-find data structure, using the fast-union with both
improvements (union by rank and path compression), to find the number of
connected components of the graph.

Note: this problem is asking you to walk through the example.  Providing
psuedocode will likely help improve your exposition, and is highly encouraged.

\nextprob
Let $n \in R$ and let $M$ be an $n \times n$ binary matrix.  Consider a
submatrix $S \subset M$; that is a rectangle-shaped contiguous set of
elements of $M$.  Let $|S|$ denote the number of elements in $S$.  We call a
submatrix \emph{fair} if it has an equal number of $1$'s and $0$'s.  Give an
algorithm to find a fair submatrix $S$ that maximizes $|S|$.

\nextprob
Let $f$ be a flow in $G=(V,E,c)$.  Prove that if $f$ is a max flow, then there
exists a cut $(S,T)$ such that $|f| = c(S,T)$.  (Note: this is the proof of $(1)
\implies (3)$ in the max-flow/min-cut theorem.  Do not prove this by proving
$(1) \implies (2) \implies (3)$.  Do the proof directly).

\nextprob
Given a weighted graph $G=(V,E,\omega)$, and two vertices $v,w \in V$, we wish
to find the \emph{shortest path} from $v$ to $w$.

\begin{enumerate}
    \item Give an algorithm to compute the shortest path.
    \item Formulate the shortest path problem as a linear program.
\end{enumerate}

\nextprob
An \emph{integer linear program} is a linear program in which the unknowns
($x_i$ from class) must be \emph{integers} instead of \emph{real numbers}.

\begin{enumerate}
    \item Is solving a linear problem or an integer linear program harder?  Why?
        (Note: you will need to consult resources to answer this question, so
        please cite your sources).
    \item Give an example of a problem that can be formulated as an integer
        linear program.  Explain why we need the integer solutions, as opposed
        to the real-valued ones.
    \item Give an example of a problem that can be formulated as a linear
        program.  Explain why the solution needs to be real numbers, not integers.
\end{enumerate}

\end{document}

