\documentclass{article}



\usepackage[margin=1in]{geometry}



\usepackage{amsfonts}

\def\R{{\mathbb R}}

\def\N{{\mathbb N}}



\title{In-Class Exercise 04}

\author{CSCI 432}



\begin{document}

\maketitle



\noindent

Group Number:\\

Group members present today:



\section*{Minimum Enclosing Disc (MED)}



\paragraph{Definitions:}



\begin{enumerate}

    \item Given $c \in \R^2$ and $r \in \R$ such that $r \geq 0$, we define the

        disc $D(c,r) := \{ x \in \R^2 ~|~ ||x-c|| \leq r \}$.

        \begin{itemize}

            \item Draw $D(0,1)$ and $D(0,2)$. Note: $0\in \R^2$ is the origin

                $(0,0)$.

                \vspace{3ex}

            \item Let $d>1$.  The generalization of a two-dimensional disc is a

                the \emph{Euclidean metric ball}.  Give a general definition of

                a ball in $\R^d$.  Denote this ball by $\mathbb{B}_d(c,r)$.

                \vspace{3ex}

            \item Draw $\mathbb{B}_1(0,1)$.

                \vspace{3ex}

        \end{itemize}

    \item A \emph{circle} is the boundary of the disc.  What is an equation that

        defines the circle $C(c,r)$?

                \vspace{3ex}

\end{enumerate}



\paragraph{Problem Statement:}



Let $P \subset \R^2$, with $|P|=n \in \N$.  We wish to find the smallest radius

$r$ such that there exists a $c \in \R^2$, where $P \subset D(c,r)$.



\begin{enumerate}

    \item If $n=1$, what is the minimum enclosing disc?  Is it unique?

        \vspace{3ex}

    \item If $n=2$, what is the minimum enclosing disc?  Is it unique?

        \vspace{3ex}

    \item If $n=3$, what is the minimum enclosing disc?  Is it unique?

        \vspace{3ex}

    \item If $n=4$, what are the possible cases that could arise?  How do we

        decide what the MED is?

        \vspace{3ex}

    \item Use the following to consider the general case:

        consider the following: choose a point $p$ at random.  Remove $p$ from $P$

        to obtain $P'$

        and compute SEB of $P'$.  What are the two cases that can happen when we

        add $p$ back in?  What is the probability of each?

        \vspace{3ex}

    \item For the expected time analysis, what is the recursion that we have?

        What is the closed form?

        \vspace{3ex}

    \item Challenge: In $\R^d$, how many points are needed in order to uniquely define a

        ball whose boundary contains those points?

        \vspace{3ex}

\end{enumerate}



\end{document}
